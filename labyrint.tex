\input def-otazky

% překlad: pdfcsplain labyrint.tex

% bordel po Játrovi:
% pokud je odpoveď moc dlohá (to dodělám sám).
%\setbox1=\vbox{\questionfont\baselineskip=50pt\hbox{uplne strasne dlouha}\hbox{odpoved}}
%\question00->23,14,11:1:Kolik zubu ma slon, slo\-nice a slune?;10;12;\lower50pt \vbox to\ht1{\box1};%


% ------------------------------------------------------------------------------
%   VYPLNIT - ZAČÁTEK
%

% tabulka použitých písmen
% n.b.: musí zůstat bez dalších nových řádků a mezer
\def\lettertable#1#2{\ifcase#1
\switchletter#2:f,G,c,P, , ;\or % 00, ..., 03
\switchletter#2:q,x,E,A, , ;\or % ...
\switchletter#2:W,m,o,j, , ;\or % ...
\switchletter#2:K,l,b,v, , ;\or % 30, ..., 33
\switchletter#2:N,e,n,Z, , ;\fi % Kompost (40), EPO (41), Cíl (42)
}

% seznam otázek
%   \questionXY->spr,sp1,sp2:permutace:Otázka?;Správná odpověď;Špatná odpověď 1;Špatná odpověď 2;%L
% kde:
%   XY          .. ID písmenka (patro + pozice)
%   spr         .. ID správné odpovědi
%   sp1/2       .. ID špatných odpovědí
%   permutace   .. náhodné číslo mezi 0-5

\question00->10,01,40:3:Ve kterém z následujících sportů se může\-me setkat s pojmem {\questionfontit karfiol}?;Divoká voda;Paragliding;Volejbal;%
\question01->11,02,40:4:Ve kterém z následujících sportů se může\-me setkat s pojmem {\questionfontit špicar}?;Lyžování;Cyklistika;Basketbal;%
\question02->12,03,40:1:Ve kterém z následujících sportů se může\-me setkat s pojmem {\questionfontit vlásenka}?;Lyžování;Paragliding;Lezení;%
\question03->13,00,40:4:Ve kterém z následujících sportů se může\-me setkat s vyjádřením {\questionfontit na kašpárka}?;Cyklistika;Basketbal;Atletika;%

%~ \setbox1=\vbox{\questionfont\baselineskip=40pt\hbox{Žádný: je to problém hardwaru.}\hbox{~}}
%~ \setbox2=\vbox{\questionfont\baselineskip=40pt\hbox{Čtyři: jeden má u sebe žárovku,}\hbox{zbylí 3 znají každý část postupu,}\hbox{jak ji našroubovat.}\hbox{~}}
%~ \setbox3=\vbox{\questionfont\baselineskip=40pt\hbox{Jeden: ale než se to naučí, spotře-}\hbox{buje spoustu dalších žárovek.}\hbox{~}}
%~ \question11->21,12,14:5:Kolik programátorů je potřeba k výměně žárovky?;{\lower40pt \vbox to\ht1{\box1}};{\lower120pt \vbox to\ht2{\box2}};{\lower80pt \vbox to\ht3{\box3}};%

\question10->20,11,12:5:Ve kterém z následujících sportů se může\-me setkat s pojmem {\questionfontit utahovák}?;Cyklistika;Kanoistika;Plavání;%
\question11->21,12,13:5:Ve kterém z následujících sportů se může\-me setkat s pojmem {\questionfontit žába}?;Lezení;Volejbal;Gymnastika;%
\question12->22,13,10:1:Ve kterém z následujících sportů se může\-me setkat s pojmem {\questionfontit ploty}?;Atletika;Basketbal;Lezení;%L
\question13->23,10,11:2:Ve kterém z následujících sportů se může\-me setkat s pojmem {\questionfontit zmrud}?;Lezení;Divoká voda;Paragliding;%

\question20->30,11,12:5:Ve kterém z následujících sportů se může\-me setkat s pojmem {\questionfontit krysa}?;Divoká voda;Plavání;Fotbal;%
\question21->31,12,13:5:Ve kterém z následujících sportů se může\-me setkat s pojmem {\questionfontit bek}?;Fotbal;Basketbal;Gymnastika;%
\question22->32,13,10:1:Ve kterém z následujících sportů se může\-me setkat s pojmem {\questionfontit fošna}?;Fotbal;Lyžování;Atletika;%L
\question23->33,10,11:2:Ve kterém z následujících sportů se může\-me setkat s pojmem {\questionfontit golfák}?;Hokej;Judo;Atletika;%

\question30->41,11,12:5:Ve kterém z následujících sportů se může\-me setkat s pojmem {\questionfontit sako}?;Volejbal;Divoká voda;Golf;%
\question31->41,12,13:5:Ve kterém z následujících sportů se může\-me setkat s pojmem {\questionfontit eprouč}?;Golf;Lezení;Judo;%
\question32->41,13,10:1:Ve kterém z následujících sportů se může\-me setkat s pojmem {\questionfontit markovátko}?;Golf;Lezení;Lyžování;%L
\question33->41,10,11:2:Ve kterém z následujících sportů se může\-me setkat s pojmem {\questionfontit hák}?;Cyklistika;Lezení;Plavání;%

\setbox1=\vbox{\questionfont\baselineskip=40pt\hbox{Ekologické řešení zpracování}\hbox{odpadů}\hbox{~}}
\setbox2=\vbox{\questionfont\baselineskip=40pt\hbox{Přehled formulářů pro finanční}\hbox{správu}\hbox{~}}
\setbox3=\vbox{\questionfont\baselineskip=40pt\hbox{Nejlepší oddíl pod širým nebem}\hbox{~}}

\question41->42,43,30:2:Která z následujících možností se {\questionfontit neskrý\-vá} pod zkratkou {\questionfontit EPO}?;{\lower80pt \vbox to\ht1{\box1}};{\lower80pt \vbox to\ht2{\box2}};{\lower40pt \vbox to\ht3{\box3}};%

% banán

%
%   VYPLNIT - KONEC
% ------------------------------------------------------------------------------


%
% Cíl
%
% vrchní písmenko
\centerline{\topletterfont\lettertable42}
% mezera
\vskip2cm
% větší řádkování
\baselineskip=50pt
% jste v cíli!
{\finishfont Gratulujeme, právě jste po\-ko\-ři\-li labyrint! Nezapomeňte o\-ra\-zit kontrolu a pokračujte dále.}
% zápatí
\footing


%
% Kompost
%
% vrchní písmenko
\centerline{\topletterfont\lettertable40}
% mezera
\vskip2cm
% alias
\centerline{\kompostfont aka.\ Kompost}
% mezera
\vskip2cm
% větší řádkování
\baselineskip=40pt
% jste na kompostě!
{\questionfont Sportovní pojmy nejsou vaše nej\-sil\-nější stránka. Nevadí, hra\-je\-me přece pro zá\-ba\-vu. Re\-cyk\-luj\-te se zpět na svoje startovní písmenko a dejte tomu ještě jed\-nu šanci!}
% zápatí
\footing


%
% Not so bad
%
% vrchní písmenko
\centerline{\topletterfont\lettertable43}
% mezera
\vskip2cm
% větší řádkování
\baselineskip=40pt
% jste na kompostě!
{\questionfont EPO je vskutku zkratka přehledu for\-mu\-lá\-řů pro finanční správu. Vraťte se na předchozí otázku a zkuste to ještě jed\-nou!}
% zápatí
\footing


%
% KONEC
%
\bye
