% ------------------------------------------------------------------------------
%   Makra pro generování permutací
%
\def\switchletter#1:#2,#3,#4,#5,#6,#7;{\ifcase#1{#2}\or{#3}\or{#4}\or{#5}\or{#6}\or{#7}\fi
}

\def\permutation#1:#2;#3;#4;#5;#6;#7;{\ifcase#1
\answerletter0:#2;#5;\answerletter1:#3;#6;\answerletter2:#4;#7;\or
\answerletter0:#3;#6;\answerletter1:#4;#7;\answerletter2:#2;#5;\or
\answerletter0:#4;#7;\answerletter1:#2;#5;\answerletter2:#3;#6;\or
\answerletter0:#2;#5;\answerletter1:#4;#7;\answerletter2:#3;#6;\or
\answerletter0:#3;#6;\answerletter1:#2;#5;\answerletter2:#4;#7;\or
\answerletter0:#4;#7;\answerletter1:#3;#6;\answerletter2:#2;#5;\fi
}

\def\answerletter#1:#2;#3;{\ifcase#1
\def\letterX{#2}\def\answerX{#3}\or
\def\letterY{#2}\def\answerY{#3}\or
\def\letterZ{#2}\def\answerZ{#3}\fi
}


% ------------------------------------------------------------------------------
%   Fonty
%
% vrchní písmenko
\font\topletterfont=csr17 at 250pt
%~ \font\topletterfont=frummr17 at 250pt   % tohle bylo nějaký runový písmo, už neumim oživit

% otázka
\font\questionfont=csr17 at 28pt

% odpovědní písmenko
%~ \font\voteletterfont=frummr17 at 22pt   % tohle bylo nějaký runový písmo, už neumim oživit
\font\voteletterfont=csr17 at 22pt

% poznámka
\font\notefont=csr17 at 20pt

% cíl
\font\finishfont=csr17 at 35pt

% kompost
\font\kompostfont=csr17 at 80pt


% ------------------------------------------------------------------------------
%   Formátování
%
\nopagenumbers
\parindent=0pt


% ------------------------------------------------------------------------------
%   Otázka
%
\def\question#1->#2,#3,#4:#5:#6;#7;#8;#9;{
    \permutation#5:\lettertable#2;\lettertable#3;\lettertable#4;#7;#8;#9;
    % větší řádkování
    \baselineskip=40pt
    % vrchní písmenko
    \centerline{\topletterfont\lettertable#1}
    % mezera
    \vskip2cm
    % otázka (6. argument)
    {\questionfont #6}
    % mezera
    \vskip1cm
    % odpovědi
    {\questionfont
        \setbox0=\hbox{\voteletterfont\letterX} \line{\hskip1cm \hskip-.5\wd0 \copy0\hskip-.5\wd0\hskip.6cm-- \answerX\hfil}
        \setbox0=\hbox{\voteletterfont\letterY} \line{\hskip1cm \hskip-.5\wd0 \copy0\hskip-.5\wd0\hskip.6cm-- \answerY\hfil}
        \setbox0=\hbox{\voteletterfont\letterZ} \line{\hskip1cm \hskip-.5\wd0 \copy0\hskip-.5\wd0\hskip.6cm-- \answerZ\hfil}
    }
    % zápatí
    \footing
}


% ------------------------------------------------------------------------------
%   Zápatí
%
\def\footing{
    \vfill
    % větší řádkování
    \baselineskip=30pt
    \centerline{\notefont Dieses Papier wird zum Spielen des Labyrinths verwendet.}
    \centerline{\notefont Bitte nehmen Sie es nicht weg. Wir werden es}
    \centerline{\notefont bis zum 5.7.2020 selbst nehmen. (Credit: Google Translate.)}
    % blíž spodnímu okraji
    \vskip-2cm
    \eject
}
