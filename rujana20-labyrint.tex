\input def-otazky

% překlad: pdfcsplain otazky.tex

% bordel po Játrovi:
% pokud je odpoveď moc dlohá (to dodělám sám).
%\setbox1=\vbox{\questionfont\baselineskip=50pt\hbox{uplne strasne dlouha}\hbox{odpoved}}
%\question00->23,14,11:1:Kolik zubu ma slon, slo\-nice a slune?;10;12;\lower50pt \vbox to\ht1{\box1};%


% ------------------------------------------------------------------------------
%   VYPLNIT - ZAČÁTEK
%

% tabulka použitých písmen
% n.b.: musí zůstat bez dalších nových řádků a mezer
\def\lettertable#1#2{\ifcase#1
\switchletter#2:f,G,c,P,Y, ;\or % 00, ..., 04
\switchletter#2:q,x,E,A,H, ;\or % ...
\switchletter#2:U,I,Z,D,s, ;\or % ...
\switchletter#2:W,m,o,j,R, ;\or % ...
\switchletter#2:K,l,b,v,T, ;\or % 40, ..., 44
\switchletter#2:N,n, , , , ;\or % Cíl (50), Kompost (51)
\switchletter#2:F,g,C,p,y, ;\fi % 60, ..., 64; 2. úroveň startu, aby se lidi ze začátku moc nepotkávali
}

% seznam otázek
%   \questionXY->spr,sp1,sp2:permutace:Otázka?;Správná odpověď;Špatná odpověď 1;Špatná odpověď 2;%L
% kde:
%   XY          .. ID písmenka (patro + pozice)
%   spr         .. ID správné odpovědi
%   sp1/2       .. ID špatných odpovědí
%   permutace   .. náhodné číslo mezi 0-5

% 51 - kompost
\question00->10,01,51:3:Jaké je protonové číslo tritia?;1;6;3;%L
\question01->11,02,51:4:Jak dlouho spí slon?;4 hodiny;2 hodiny;12 hodin;%
\question02->12,03,51:1:Sloveso nanosit se je:;Dokonavé;Nedokonavé;Obouvidové;%L
\question03->13,04,51:4:How much miskates in these sentence?;4;2;3;%
\question04->14,00,51:3:Jaký je RGB kód oranžové?;255, 128, 0;255, 255, 0;128, 255, 0;%

\question60->10,61,51:3:Jaké je protonové číslo tritia?;1;6;3;%L
\question61->11,62,51:4:Jak dlouho spí slon?;4 hodiny;2 hodiny;12 hodin;%
\question62->12,63,51:1:Sloveso nanosit se je:;Dokonavé;Nedokonavé;Obouvidové;%L
\question63->13,64,51:4:How much miskates in these sentence?;4;2;3;%
\question64->14,60,51:3:Jaký je RGB kód oranžové?;255, 128, 0;255, 255, 0;128, 255, 0;%

\question10->20,11,13:5:Jaký byl nejvyšší oficiální denní přírus\-tek případů koronaviru v Německu?;6 933;11 218;3 102;%
\setbox1=\vbox{\questionfont\baselineskip=40pt\hbox{Žádný: je to problém hardwaru.}\hbox{~}}
\setbox2=\vbox{\questionfont\baselineskip=40pt\hbox{Čtyři: jeden má u sebe žárovku,}\hbox{zbylí 3 znají každý část postupu,}\hbox{jak ji našroubovat.}\hbox{~}}
\setbox3=\vbox{\questionfont\baselineskip=40pt\hbox{Jeden: ale než se to naučí, spotře-}\hbox{buje spoustu dalších žárovek.}\hbox{~}}
\question11->21,12,14:5:Kolik programátorů je potřeba k výměně žárovky?;{\lower40pt \vbox to\ht1{\box1}};{\lower120pt \vbox to\ht2{\box2}};{\lower80pt \vbox to\ht3{\box3}};%
\question12->22,13,10:5:Jaký je současný rychlostní rekord na windsurfu?;52 uzlů;34 m/s;73 km/h;%
\question13->23,14,11:1:Kolikátý je letošní ročník ws kurzu na Rujáně?;7.;6.;5.;%L
\question14->24,10,12:2:Jak dlouho trvá nejdelší den ve Wie\-ku?;17:17;16:23;16:48;%

\question20->30,21,23:5:Jaká kyselina je vitamin C?;L-askorbová;D-askorbová;R-askorbová;%
\question21->31,22,24:1:Siam je starší název pro:;Thajsko;Vietnam;Tchaj-wan;%
\question22->32,23,20:2:Havárie jaderné elektrárny v Černoby\-lu se stala:;26.\ dubna 1986;1.\ května 1982;23.\ dubna 1985;%L
\question23->33,24,21:0:Kolik nejvíce koňů lze umístit na ša\-cho\-vni\-ci osm krát osm tak, aby se žád\-ní dva neohrožovali?;32;24;16;%
\question24->34,20,22:3:Co je to žžonka;Alkoholický nápoj;Blbost;Čečenská prostitutka;%L

\question30->40,31,33:3:Jaká je nejvyšší hora Šumavy?;Velký Javor;Plechý;Velký Roklan;%L
\question31->41,32,34:0:Čím se liší travní flosna od normální?;Je zkosená;Má v sobě otvor;Je z karbonizovaného bambusu;%
\question32->42,33,30:5:Které z následujících zvířat není o\-brat\-lo\-vec?;Rak;Chřestýš;Žralok;%L
\question33->43,34,31:2:Jaký akord je tvořen tóny D, F, A a C?;Dmi7;A4;Amaj7;%L
\question34->44,30,32:4:Kolik dětí měl Johann Sebastian Bach?;20;0;11;%

% 50 - cíl
\question40->50,41,43:5:Jak jste se dostali k této otázce?;Orgové jsou strašně škodolibí;Jsme lamy;Rádi bloudíme;%
\question41->50,42,44:2:Mistrovství světa v softbalu se v roce 2019 konalo v Praze a?;Havlíčkově Brodě;Kolíně;Klatovech;%L
\question42->50,43,40:4:Co se děje při spin-outu?;Za flosnou se drží vzduchová kapsa;Prkno vyskakuje nekontrolovaně z vody;Vylejzá ti zadek z plavek;%
\question43->50,44,41:1:Kolik poštovních známek "Modrý mauricius" bylo celkově vyrobeno?;500;5 000;50 000;%
\question44->50,40,42:4:V roce 2010 vybuchla na Islandu sopka, která nadvakrát ochromila leteckou dopravu nad Evropou. Jak se jmenuje stej\-nojmenný ledovec, pod kterým se vul\-kán nachází?;Eyjafjallajökull;Eiríksjökull;Vatnajökull;%L
%
%   VYPLNIT - KONEC
% ------------------------------------------------------------------------------


%
% Cíl
%
% vrchní písmenko
\centerline{\topletterfont\lettertable50}
% mezera
\vskip2cm
% větší řádkování
\baselineskip=55pt
% jste v cíli!
{\finishfont\hskip1.8cm Gratulujeme! Jste opravdoví drs\-ňá\-ci, protože jste právě došli do cíle. Nebylo to snadné, že? Ví\-těz\-ný pocit si můžete vychutnávat cestou do objektu. Těšíme se na vás!!}
% zápatí
\footing


%
% Kompost
%
% vrchní písmenko
\centerline{\topletterfont\lettertable51}
% mezera
\vskip2cm
% alias
\centerline{\kompostfont aka.\ Kompost}
% mezera
\vskip2cm
% větší řádkování
\baselineskip=55pt
% jste na kompostě!
\centerline{\finishfont Recyklujte se zpět na start.}
% zápatí
\footing


%
% KONEC
%
\bye
